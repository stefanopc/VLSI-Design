\section{SMT}

Another approach to this problem involves SMT (Satisfiability Modulo Theories) with Z3Py, that was chosen because of its expressive power (due to the use of first-order logic) compared to SAT, which instead can be complex to formalize. The main drawback of SMT models is the worse efficiency, which however is compensated by the higher expressivity.

\subsection{Variables}

Modeling with SMT followed the same scheme that was seen for CP.
First, the coordinates of the circuits have been encoded in two vectors $x$ and $y$ of integers with dimension given by the number of circuits.
As previously seen, the arrays $x\_dim$ and $y\_dim$ contain the widths and heights of all the circuits, while $height$ represents the maximum height of the plate.


As done before, the variable domain has been reduced to speed up the search: first of all, the horizontal coordinates $x_i$ have to be greater than zero and their sum with the circuit widths must be below the plate width.

\begin{equation*}
    \forall i \in \{1..n\} \quad x_i \geq 0 \; \land x_i + x\_dim_i \leq w
\end{equation*}

Similarly, the vertical coordinates $y_i$ have to be greater than zero and their sum with the circuit heights must be below the maximum plate height.

\begin{equation*}
     \forall i \in \{1..n\} \quad y_i \geq 0 \; \land y_i + y\_dim_i \leq height
\end{equation*}

\subsection{Objective function}

The \textbf {objective function} and its bounds are the ones which were previously described in section 2.2.

\subsection{Constraints}

The main constraints are meant to tell the solver to respect the boundaries of the plate for the horizontal and vertical coordinates:

\begin{equation*}
     \forall i \in \{1..n\} \quad x_i + x\_dim_i \leq w
\end{equation*}
\begin{equation*}
     \forall i \in \{1..n\} \quad y_i + y\_dim_i \leq height
\end{equation*}

Furthermore, a similar definition of the global constraint \textit{cumulative} (previously used in Minizinc for the CP solution) was given, with the same reasoning that was described before.

This is defined in the following way:
\begin{gather*}
cumulative(y, y\_dim, x\_dim, width) \\
cumulative(x, x\_dim, y\_dim, height)
\end{gather*}

Moreover, a non-overlapping constraint is necessary to make sure that no pair of rectangles overlap:
\begin{equation*}
     \forall i,j \in \{1..n\}, i \neq j \quad (x_i + x\_dim_i \leq x_j) \lor (x_j + x\_dim_j \leq x_i) \lor (y_i + y\_dim_i \leq y_j) \lor (y_i + y_j + y\_dim_j \leq y_i)
\end{equation*}

Symmetry braking constraints are the same as in CP: it has been decided to put the biggest circuit in the bottom left part of the plate, at x and y coordinates equal to zero.

\subsection{Rotation model}

As seen in CP, for the rotation model we introduce two other vectors $x\_dim\_rot$ and $y\_dim\_rot$ which contain the actual widths and heights considered, since in case of rotation the values of width and height will be swapped. 
\begin{gather*}
     \forall i \in \{1..n\} \quad (x\_dim\_rot_i = x\_dim_i \; \land \; y\_dim\_rot_i = y\_dim_i) \; \lor \; \\
     \lor \; (x\_dim\_rot_i = y\_dim_i \land y\_dim\_rot_i = x\_dim_i)
\end{gather*}

As previously seen, we add a constraint for those cases where a circuit has the same horizontal and vertical dimensions, hence forcing it to avoid rotation:
\begin{equation*}
     \forall i \in \{1..n\} \quad x\_dim_i = y\_dim_i \implies (x\_dim\_rot_i = x\_dim_i \land y\_dim\_rot_i = y\_dim_i)
\end{equation*}

As previously done, the output file is modified to print the string \textit{"rotated"} in case a circuit is rotated. 

\subsection{Validation}

The standard search method provided by Z3Py was used to compare the performance of the models: the class \verb|Optimize| allows to get an assignment for each variable which minimizes the objective function through the method \verb|Optimize.minimize(<minimization objective>)|, whose aim is to minimize the objective variable \textit{height}.

\hfill 

\begin{center}
\begin{longtable}{|l|l|l|l|l|l|}
\caption{Solving times and height obtained by SMT models (base and rotation) with and without symmetry braking constraints} \label{tab:long} \\

\hline \multicolumn{1}{|c|}{\textbf{Instance}} & \multicolumn{1}{c|}{\textbf{Base + SB}} & \multicolumn{1}{c|}{\textbf{Base w/o SB}} & \multicolumn{1}{c|}{\textbf{Rot + SB}} & \multicolumn{1}{c|}{\textbf{Rot w/o SB}} & \multicolumn{1}{c|}{\textbf{Height}} \\ \hline 
\endfirsthead

\multicolumn{6}{c}
{{\bfseries \tablename\ \thetable{} -- continued from previous page}} \\
\hline \multicolumn{1}{|c|}{\textbf{Instance}} & \multicolumn{1}{c|}{\textbf{Base w/ SB}} & \multicolumn{1}{c|}{\textbf{Base w/o SB}} & \multicolumn{1}{c|}{\textbf{Rot w/ SB}} & \multicolumn{1}{c|}{\textbf{Rot w/o SB}} & \multicolumn{1}{c|}{\textbf{Height}} \\ \hline 
\endhead

\hline \multicolumn{6}{|r|}{{Continued on next page}} \\ \hline
\endfoot

\hline \hline
\endlastfoot

1 & 0,005 & 0,025 & 0,017 & 0,032 & 8 \\
2 & 0,010 & 0,088 & 0,048 & 0,065 & 9 \\
3 & 0,014 & 0,010 & 0,084 & 0,124 & 10 \\
4 & 0,024 & 0,039 & 0,286 & 0,631 & 11 \\
5 & 0,040 & 0,073 & 0,782 & 0,709 & 12 \\
6 & 0,061 & 0,129 & 0,539 & 2,751 & 13 \\
7 & 0,078 & 0,113 & 0,808 & 2,567 & 14 \\
8 & 0,091 & 1,127 & 1,282 & 0,515 & 15 \\
9 & 0,102 & 0,179 & 1,100 & 3,558 & 16 \\
10 & 0,321 & 0,417 & 44,76 & 137,9 & 17 \\
11 & - & - & - & - & - \\
12 & 0,982 & 1,443 & 71,52 & 26,48 & 19 \\
13 & 0,916 & 1,809 & 41,35 & 122,9 & 20 \\
14 & 2,814 & 3,896 & 107,4 & - & 21 \\
15 & 1,166 & 2,115 & 65,36 & 243,9 & 22 \\
16 & - & - & - & - & - \\
17 & 5,682 & 7,712 & 226,4 & 112,9 & 24 \\
18 & 5,793 & 8,449 & - & - & 25 \\
19 & - & - & - & - & - \\
20 & 145,0 & - & - & - & 27 \\
21 & - & - & - & - & - \\
22 & - & - & - & - & - \\
23 & 16,75 & 23,65 & - & - & 30 \\
24 & 12,27 & 11,38 & - & - & 31 \\
25 & - & - & - & - & - \\
26 & 94,15 & 117,6 & - & - & 33 \\
27 & 21,21 & 21,56 & - & - & 34 \\
28 & 33,05 & 40,65 & - & - & 35 \\
29 & 60,46 & 67,93 & - & - & 36 \\
30 & - & - & - & - & - \\
31 & 7,358 & 11,84 & 238,7 & - & 38 \\
32 & - & - & - & - & - \\
33 & 18,65 & 17,31 & 291,4 & 272,0 & 40 \\
34 & - & - & - & - & - \\
35 & - & - & - & - & - \\
36 & - & - & - & - & - \\
37 & - & - & - & - & - \\
38 & - & - & - & - & - \\
39 & - & - & - & - & - \\
40 & - & - & - & - & - \\
\end{longtable}
\end{center}

\newpage