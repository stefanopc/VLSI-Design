\section{Introduction}

VLSI (Very Large Scale Integration) refers to the trend of integrating circuits into silicon chips.

Given a fixed-width plate and a list of rectangular circuits, the problem consists in deciding how to place the circuits on the plate in order to minimize the length of the final device. The combinatorial optimization problem will be modeled and solved using two approaches: Constraint Programming (CP) and Satisfiability Modulo Theories (SMT). In both cases, in addition to a base model, it will be developed a model which takes into account the rotation of the circuits, in order to see what modifications should be applied to the models and to compare the solutions and their complexity.

The input consists of a set of 40 instances in a text format, each one containing the following data:

\begin{itemize}
    \item width of the silicon plate $w$
    \item number of circuits to place $n$
    \item horizontal dimension $x_i$ of the i-th circuit (for each circuit i from 1..n)
    \item vertical dimension $y_i$ of the i-th circuit (for each circuit i from 1..n)
\end{itemize}\

The output, instead, will be structured in the following way:

$ w \quad h \\
n \\
x_0\quad y_0\quad \hat{x_0}\quad \hat{y_0} \\
x_1\quad y_0\quad \hat{x_1}\quad \hat{y_1} \\
. . . \\
x_n\quad y_n\quad \hat{x_n}\quad \hat{y_n} \\
$

where $h$ is the maximum height reached by the circuits configuration; $x_i$ and $y_i$ are respectively the horizontal and vertical dimensions of the i-th circuit; $\hat{x_i}$ and $\hat{y_i}$ are the coordinates of the left-bottom corner for the i-th circuit. The goal is to minimize the final height of the device.

The parameters used to formalize the problem in all the models are the following:

\begin{itemize}
    \item $width$ is the plate width
    \item $n$ is the number of circuits to place in the plate
    \item $x\_dim$ and $y\_dim$ are two arrays with the horizontal and vertical sizes of the circuits, respectively (indexed from 1..n)
\end{itemize}\


